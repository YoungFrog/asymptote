\documentclass[a4paper]{ltxdoc}
\usepackage{hyperref}
%\usepackage{url}

\begin{document}

%\GetFileInfo{pdfanim.sty}

\title{The \textsf{PDFAnim} package}
\author{Jochen Skupin (modified by John Bowman)}
\date{\today}
\maketitle

\begin{abstract}
  A \LaTeX\ package to create animated PDFs with pdf\LaTeX.
\end{abstract}

\changes{v0.03}
	{2003/03/24}{First animation for \TeX -Showcase: \protect\url{http://www.tug.org/texshowcase/}}
\changes{v0.51}
	{2004/06/13}{First published version}
\changes{v0.53}
	{2006/11/26}{Second published version}

%\clearpage
\vfill
\tableofcontents
\clearpage
 
\section{Introduction}
  PDFAnim is a \LaTeX\ package to create animated PDFs with pdf\LaTeX.
  
  As it is very unusual to create animations with \LaTeX
  and to have to deal with JavaScript it might be that large
  parts of this document are not understandable for the
  normal \LaTeX user. Don't be too frustrated about it and
  just try the examples.
  
  Comments and bug reports are {\em very} welcome. 
  You can contact me at:
  \begin{center}
  \href{mailto:jochen.skupin@uni-bremen.de}{jochen.skupin@uni-bremen.de}
  \end{center}
  
  Don't be too disappointed if it takes a while until
  you get an answer, as I have little spare time to 
  work on this package.


\section{Usage}
To include an animation in a PDF file you need:
\begin{itemize}
\item PDF\LaTeX
\item the PDFAnim package (i.e. the |pdfanim.sty|-file)
\item The frames of the animation as single PDF files
  with filenames in the form
  |filename|\textit{n}|.pdf|
  where \textit{n} is the number of the \textit{n}th frame of
  the animation starting with 0.
\item Adobe Acrobat or AcroReader to view the animation
\end{itemize}

The package includes the single frames as hidden picture buttons in the
PDF-file. The animation itself is created using a visible picture
button whose picture is exchanged using the JavaScript language available
in Adobe Acrobat and AcroReader.
The frames of the animation have to be included in the document preamble using
\begin{center}
  |\PDFAnimLoad[options]{name}{filename}{number_of_frames}|
\end{center}
where |filename| is the head of the filename of the PDF-files
containing the single frames of the animation omitting the running number
and the |.pdf| extension.
|name| is the name of the animation. With this name it can be accessed
in the document. |number_of_frames| gives the number of files to be
included. An animation with 10 frames would require 
|number_of_frames| $=10$ and files |filename0...9.pdf|.
The various |options| of |\PDFAnimLoad| will be discussed
in section \ref{sec:options}.

\noindent
To access the animation anywhere in the document use:
\begin{center}
  |\PDFAnimation{name}|
\end{center}

\noindent
Examples can be found at:
\begin{itemize}
\item
\url{http://www.uni-bremen.de/~skupin/pdfanim/Demos}
\item
\url{http://www.tug.org/texshowcase/}
\end{itemize}


\section{Package option(s)}
\label{sec:pkgoptions}

The package has to be loaded in the document preamble with:
\begin{center}
  |\usepackage[options]{pdfanim}|
\end{center}
The package |options| change the way the JavaScript
code is included in the PDF-file.

\begin{macro}{default}
The default behaviour (by giving no options at all)
is to include most of the JavaScript on document level
and access it via page open/close attributes or via the on-click
action of the picture button. 
\begin{itemize}
\item
The advantage is that only very short
function calls have to be included in the page open/close attributes
because most JavaScript code is stored on document level.
Also it is possible to define global variables to store the
status of an animation (like the last displayed frame, is it running or not). 
\item
\end{itemize}
\end{macro}

\begin{macro}{NoDocJS}
With this option no document level JavaScript is generated.
All JavaScript is stored in the picture button itself and 
the page open/close attributes.
\begin{itemize}
\item
The disadvantage is that all JavaScript code is stored on every page
again. This leads to slightly bigger file sizes.
Due to this all variables are newly initialised on every page
so it is not possible to have global variables to remember the status
of an animation from page to page.
\end{itemize}
\end{macro}

\begin{macro}{NoPageJS}
I didn't yet manage to derive on which page a distinct animation
(especially floats) is placed by \TeX. The JavaScript code
for all animations is therefore included in every page. Not very 
elegant---I know. The problem is that when the document contains
auto starting animations all of them are started again on {\em every} 
page. With the |NoPageJS| option the automatic inclusion of
JavaScript code on any page can be suppressed. 
\begin{itemize}
\item
The advantage is that not all auto start animations will run permanently.
\item
The disadvantage is that the user has to take care to put the 
JavaScript on pages with animations by hand. To do this the
following 4 commands are provided:
\end{itemize}
\end{macro}

\begin{macro}{\PDFAnimJSPageEnable}
enable JavaScript in page-attribute of current page
\end{macro}
\begin{macro}{\PDFAnimJSEnable}
enable JavaScript in page-attribute on all following pages
\end{macro}
\begin{macro}{\PDFAnimJSPageDisable}
disable JavaScript in page-attribute of current page
\end{macro}
\begin{macro}{\PDFAnimJSDisable}
disable JavaScript in page-attribute on all following pages
\end{macro}


\section{\textsf{PDFAnimLoad} options}
\label{sec:options}

There are already much too many options. But during my usage 
of PDFAnim I needed all of them for one or the other reason.
Unfortunately not all options work well together and some of the
older ones might not work at all in the current implementation
anymore. Maybe some clean-up is needed here.

\subsection{Boolean options}
\label{sec:booloptions}

\begin{macro}{single}
extract frames as the individual pages of a single input file
\end{macro}

\begin{macro}{auto}
auto start animation
\end{macro}

\begin{macro}{debug}
enable debug messages
\end{macro}

\begin{macro}{fallback}
include start-picture below PictureButton
as fallback solution for xpdf, macs ...
(gives poor animation results)
\end{macro}

\begin{macro}{hidden}
create hidden PictureButton
(mostly for internal use)
\end{macro}

\begin{macro}{loop}
loop animation
\end{macro}

\begin{macro}{noclick}
don't recognise clicks on PictureButton
\end{macro}

\begin{macro}{remember}
remember last displayed picture when
changing to another page
\end{macro}

\begin{macro}{reverse}
play animation in reversed order
\end{macro}

\begin{macro}{step}
advance animation on every mouse-click
\end{macro}

\subsection{Options that take parameters}
\label{sec:paroptions}

\begin{macro}{bcolor}
bordercolor of PictureButton
\end{macro}

\begin{macro}{bgcolor}
backgroundcolor of PictureButton
\end{macro}

\begin{macro}{defaultframe}
select frame to display when animation not yet running
\end{macro}

\begin{macro}{depth}
depth of PictureButton
\end{macro}

\begin{macro}{extension}
set extension of included pictures
(till now only pdf works)
\end{macro}

\begin{macro}{height}
height of PictureButton
\end{macro}

\begin{macro}{interval}
set interval in ms between animation frames
(only shows effect if interval is longer than
the time needed to display a frame)
\end{macro}

\begin{macro}{name}
name PictureButton
(may be useful for further editing of the pdf,
not needed/used by PDFAnim)
\end{macro}

\begin{macro}{onclick}
JavaScript action to perform on mouse-click
(mostly for internal use)
\end{macro}

\begin{macro}{scale}
scaling of picture used in PictureButton:
\begin{itemize}
\item[A] Always scale.
\item[B] Scale only when the icon is bigger than the annotation rectangle.
\item[S] Scale only when the icon is smaller than the annotation rectangle.
\item[N] Never scale.
\end{itemize}
\end{macro}

\begin{macro}{scaletype}
type of scaling of picture used in PictureButton:
\begin{itemize}
\item[A] 
  Anamorphic scaling: scale the icon to the annotation rectangle
  exactly, without regard to its original aspect ratio (ratio of width to
  height).
\item[P]
  Proportional scaling: scale the icon to the width or
  height of the annotation rectangle while maintaining the icon's
  original aspect ratio. If the required horizontal and vertical scaling
  factors are different, use the smaller of the two, centering the icon
  within the annotation rectangle in the other
  dimension.
\end{itemize}
\end{macro}

\begin{macro}{use}
use pictures from other animation
(to save memory)
\end{macro}

\begin{macro}{usecnt}
use counter from other animation
\end{macro}

\begin{macro}{startframe}
select first frame to display
\end{macro}

\begin{macro}{width}
width of PictureButton
\end{macro}

\subsection{Navigation buttons}
Another command, |\AnimButton|, which is available if you also load
D. P. Story's package |eforms.sty|, presents an alternative method for starting
and stoping animations by pressing the Play/Pause button. It also provides
a method for jumping to arbitrary locations in long animations. I the
animation is not running, you can input a percentage; pressing
Play/Pause then starts the animation from that relative position (e.g.\ 75
means frame 150 for a 200-frame animation).

\section{Open issues}
\subsection*{Fewer options}
During my usage of PDFAnim I used all of the options 
in section \ref{sec:options}. But I guess they can be reduced and combined
in a better way. Any improvements are welcome.

\end{document}
