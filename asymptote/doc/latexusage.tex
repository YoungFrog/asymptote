\documentclass[12pt]{article}

% Use this form to include eps (latex) or pdf (pdflatex) files:
\usepackage{asymptote}

% Use this form with latex or pdflatex to include inline LaTeX code:
%\usepackage[inline]{asymptote}

% Enable this line to produce pdf hyperlinks with latex:
%\usepackage[hypertex]{hyperref}

% Enable this line to produce pdf hyperlinks with pdflatex:
%\usepackage[pdftex]{hyperref}

\begin{document}

\begin{asydef}
// Global Asymptote definitions can be put here.
usepackage("bm");
\end{asydef}

Here is a venn diagram produced with Asymptote, drawn to width 4cm:

\def\A{A}
\def\B{\bm{B}}

%\begin{figure}
\begin{center}
\begin{asy}
size(4cm,0);
pen colour1=red;
pen colour2=green;

pair z0=(0,0);
pair z1=(-1,0);
pair z2=(1,0);
real r=1.5;
path c1=circle(z1,r);
path c2=circle(z2,r);
fill(c1,colour1);
fill(c2,colour2);

picture intersection=new picture;
fill(intersection,c1,colour1+colour2);
clip(intersection,c2);

add(intersection);

draw(c1);
draw(c2);

//draw("$\A$",box,z1);              // Requires [inline] package option.
//draw(Label("$\B$","$B$"),box,z2); // Requires [inline] package option.
draw("$A$",box,z1);            
draw("$\bm{B}$",box,z2);

pair z=(0,-2);
real m=3;
margin BigMargin=Margin(0,m*dot(unit(z1-z),unit(z0-z)));

draw(Label("$A\cap B$",0),conj(z)--z0,Arrow,BigMargin);
draw(Label("$A\cup B$",0),z--z0,Arrow,BigMargin);
draw(z--z1,Arrow,Margin(0,m));
draw(z--z2,Arrow,Margin(0,m));

shipout(bbox(0.25cm));
\end{asy}
%\caption{Venn diagram}\label{venn}
\end{center}
%\end{figure}

Each graph is drawn in its own environment. One can specify the width
and height to \LaTeX\ explicitly:

\begin{center}
\begin{asy}[3cm,0]
guide center = (0,1){W}..tension 0.8..(0,0){(1,-.5)}..tension 0.8..{W}(0,-1); 

draw((0,1)..(-1,0)..(0,-1));
filldraw(center{E}..{N}(1,0)..{W}cycle);
fill(circle((0,0.5),0.125),white);
fill(circle((0,-0.5),0.125));
\end{asy}
\end{center}

One can also scale the figure to the full line width:
\begin{center}
\begin{asy}[\the\linewidth]
settings.thick=false;
if(settings.render < 0) settings.render=4;

import graph3;
import solids;

currentprojection=perspective(0,1,30,up=Y);
currentlight=light(gray(0.75),specularfactor=3,viewport=false,
                   (0.25,0.5,1),(0,-0.5,-0.5));

pen color=green;
real alpha=240;

real f(real x) {return 2x^2-x^3;}
pair F(real x) {return (x,f(x));}
triple F3(real x) {return (x,f(x),0);}

int n=10;
path3[] blocks=new path3[n];
for(int i=1; i <= n; ++i) {
  real height=f((i-0.5)*2/n);
  real left=(i-1)*2/n;
  real right=i*2/n;
  blocks[i-1]=
    (left,0,0)--(left,height,0)--(right,height,0)--(right,0,0)--cycle;
}

path p=graph(F,0,2,n,operator ..)--cycle;
surface s=surface(bezulate(p));
path3 p3=path3(p);

revolution a=revolution(p3,Y,0,alpha);
draw(surface(a),color);
draw(rotate(alpha,Y)*s,color);
for(int i=0; i < n; ++i)
  draw(surface(blocks[i]),color+opacity(0.5),black);
draw(p3);

xaxis3(Label("$x$",1,align=2X),Arrow3);
yaxis3(Label("$y$",1,align=2Y),ymax=1.3,dashed,Arrow3);
arrow("$y=2x^2-x^3$",XYplane(F(1.8)),X+Z,1.5cm,red);
draw(arc(1.17Y,0.3,90,0,7.5,180),Arrow3);
\end{asy}
\end{center}

\end{document}
